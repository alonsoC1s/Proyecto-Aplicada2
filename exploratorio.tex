% Options for packages loaded elsewhere
\PassOptionsToPackage{unicode}{hyperref}
\PassOptionsToPackage{hyphens}{url}
%
\documentclass[
]{article}
\usepackage{amsmath,amssymb}
\usepackage{lmodern}
\usepackage{iftex}
\ifPDFTeX
  \usepackage[T1]{fontenc}
  \usepackage[utf8]{inputenc}
  \usepackage{textcomp} % provide euro and other symbols
\else % if luatex or xetex
  \usepackage{unicode-math}
  \defaultfontfeatures{Scale=MatchLowercase}
  \defaultfontfeatures[\rmfamily]{Ligatures=TeX,Scale=1}
\fi
% Use upquote if available, for straight quotes in verbatim environments
\IfFileExists{upquote.sty}{\usepackage{upquote}}{}
\IfFileExists{microtype.sty}{% use microtype if available
  \usepackage[]{microtype}
  \UseMicrotypeSet[protrusion]{basicmath} % disable protrusion for tt fonts
}{}
\makeatletter
\@ifundefined{KOMAClassName}{% if non-KOMA class
  \IfFileExists{parskip.sty}{%
    \usepackage{parskip}
  }{% else
    \setlength{\parindent}{0pt}
    \setlength{\parskip}{6pt plus 2pt minus 1pt}}
}{% if KOMA class
  \KOMAoptions{parskip=half}}
\makeatother
\usepackage{xcolor}
\IfFileExists{xurl.sty}{\usepackage{xurl}}{} % add URL line breaks if available
\IfFileExists{bookmark.sty}{\usepackage{bookmark}}{\usepackage{hyperref}}
\hypersetup{
  pdftitle={Exploración de variables y el modelo.},
  pdfauthor={Alonso Martinez; Enrique Ampudia; Ana Tena; Rodrigo Peralta; Cecilia Ramirez},
  hidelinks,
  pdfcreator={LaTeX via pandoc}}
\urlstyle{same} % disable monospaced font for URLs
\usepackage[margin=1in]{geometry}
\usepackage{color}
\usepackage{fancyvrb}
\newcommand{\VerbBar}{|}
\newcommand{\VERB}{\Verb[commandchars=\\\{\}]}
\DefineVerbatimEnvironment{Highlighting}{Verbatim}{commandchars=\\\{\}}
% Add ',fontsize=\small' for more characters per line
\usepackage{framed}
\definecolor{shadecolor}{RGB}{248,248,248}
\newenvironment{Shaded}{\begin{snugshade}}{\end{snugshade}}
\newcommand{\AlertTok}[1]{\textcolor[rgb]{0.94,0.16,0.16}{#1}}
\newcommand{\AnnotationTok}[1]{\textcolor[rgb]{0.56,0.35,0.01}{\textbf{\textit{#1}}}}
\newcommand{\AttributeTok}[1]{\textcolor[rgb]{0.77,0.63,0.00}{#1}}
\newcommand{\BaseNTok}[1]{\textcolor[rgb]{0.00,0.00,0.81}{#1}}
\newcommand{\BuiltInTok}[1]{#1}
\newcommand{\CharTok}[1]{\textcolor[rgb]{0.31,0.60,0.02}{#1}}
\newcommand{\CommentTok}[1]{\textcolor[rgb]{0.56,0.35,0.01}{\textit{#1}}}
\newcommand{\CommentVarTok}[1]{\textcolor[rgb]{0.56,0.35,0.01}{\textbf{\textit{#1}}}}
\newcommand{\ConstantTok}[1]{\textcolor[rgb]{0.00,0.00,0.00}{#1}}
\newcommand{\ControlFlowTok}[1]{\textcolor[rgb]{0.13,0.29,0.53}{\textbf{#1}}}
\newcommand{\DataTypeTok}[1]{\textcolor[rgb]{0.13,0.29,0.53}{#1}}
\newcommand{\DecValTok}[1]{\textcolor[rgb]{0.00,0.00,0.81}{#1}}
\newcommand{\DocumentationTok}[1]{\textcolor[rgb]{0.56,0.35,0.01}{\textbf{\textit{#1}}}}
\newcommand{\ErrorTok}[1]{\textcolor[rgb]{0.64,0.00,0.00}{\textbf{#1}}}
\newcommand{\ExtensionTok}[1]{#1}
\newcommand{\FloatTok}[1]{\textcolor[rgb]{0.00,0.00,0.81}{#1}}
\newcommand{\FunctionTok}[1]{\textcolor[rgb]{0.00,0.00,0.00}{#1}}
\newcommand{\ImportTok}[1]{#1}
\newcommand{\InformationTok}[1]{\textcolor[rgb]{0.56,0.35,0.01}{\textbf{\textit{#1}}}}
\newcommand{\KeywordTok}[1]{\textcolor[rgb]{0.13,0.29,0.53}{\textbf{#1}}}
\newcommand{\NormalTok}[1]{#1}
\newcommand{\OperatorTok}[1]{\textcolor[rgb]{0.81,0.36,0.00}{\textbf{#1}}}
\newcommand{\OtherTok}[1]{\textcolor[rgb]{0.56,0.35,0.01}{#1}}
\newcommand{\PreprocessorTok}[1]{\textcolor[rgb]{0.56,0.35,0.01}{\textit{#1}}}
\newcommand{\RegionMarkerTok}[1]{#1}
\newcommand{\SpecialCharTok}[1]{\textcolor[rgb]{0.00,0.00,0.00}{#1}}
\newcommand{\SpecialStringTok}[1]{\textcolor[rgb]{0.31,0.60,0.02}{#1}}
\newcommand{\StringTok}[1]{\textcolor[rgb]{0.31,0.60,0.02}{#1}}
\newcommand{\VariableTok}[1]{\textcolor[rgb]{0.00,0.00,0.00}{#1}}
\newcommand{\VerbatimStringTok}[1]{\textcolor[rgb]{0.31,0.60,0.02}{#1}}
\newcommand{\WarningTok}[1]{\textcolor[rgb]{0.56,0.35,0.01}{\textbf{\textit{#1}}}}
\usepackage{graphicx}
\makeatletter
\def\maxwidth{\ifdim\Gin@nat@width>\linewidth\linewidth\else\Gin@nat@width\fi}
\def\maxheight{\ifdim\Gin@nat@height>\textheight\textheight\else\Gin@nat@height\fi}
\makeatother
% Scale images if necessary, so that they will not overflow the page
% margins by default, and it is still possible to overwrite the defaults
% using explicit options in \includegraphics[width, height, ...]{}
\setkeys{Gin}{width=\maxwidth,height=\maxheight,keepaspectratio}
% Set default figure placement to htbp
\makeatletter
\def\fps@figure{htbp}
\makeatother
\setlength{\emergencystretch}{3em} % prevent overfull lines
\providecommand{\tightlist}{%
  \setlength{\itemsep}{0pt}\setlength{\parskip}{0pt}}
\setcounter{secnumdepth}{-\maxdimen} % remove section numbering
\ifLuaTeX
  \usepackage{selnolig}  % disable illegal ligatures
\fi

\title{Exploración de variables y el modelo.}
\author{Alonso Martinez \and Enrique Ampudia \and Ana Tena \and Rodrigo
Peralta \and Cecilia Ramirez}
\date{2022-05-10}

\begin{document}
\maketitle

\hypertarget{anuxe1lisis-exploratorio}{%
\section{Análisis exploratorio}\label{anuxe1lisis-exploratorio}}

\hypertarget{autos-promedio-de-co-a-travuxe9s-del-tiempo.}{%
\subsection{Autos \& promedio de CO a través del
tiempo.}\label{autos-promedio-de-co-a-travuxe9s-del-tiempo.}}

\begin{verbatim}
## `geom_smooth()` using formula 'y ~ x'
\end{verbatim}

\includegraphics{exploratorio_files/figure-latex/unnamed-chunk-2-1.pdf}

\hypertarget{autos-promedio-de-co-a-travuxe9s-del-tiempo.-1}{%
\subsection{Autos \& promedio de CO a través del
tiempo.}\label{autos-promedio-de-co-a-travuxe9s-del-tiempo.-1}}

\begin{verbatim}
## `geom_smooth()` using formula 'y ~ x'
\end{verbatim}

\includegraphics{exploratorio_files/figure-latex/unnamed-chunk-3-1.pdf}

\hypertarget{autos-promedio-de-co-a-travuxe9s-del-tiempo.-2}{%
\subsection{Autos \& promedio de CO a través del
tiempo.}\label{autos-promedio-de-co-a-travuxe9s-del-tiempo.-2}}

\begin{verbatim}
## `geom_smooth()` using formula 'y ~ x'
\end{verbatim}

\includegraphics{exploratorio_files/figure-latex/unnamed-chunk-4-1.pdf}

\hypertarget{nuxfamero-de-estaciones-de-de-transporte-puxfablico-eluxe9ctrico-a-travuxe9s-del-tiempo}{%
\subsection{Número de estaciones de de transporte público eléctrico a
través del
tiempo}\label{nuxfamero-de-estaciones-de-de-transporte-puxfablico-eluxe9ctrico-a-travuxe9s-del-tiempo}}

\begin{verbatim}
## `geom_smooth()` using formula 'y ~ x'
\end{verbatim}

\includegraphics{exploratorio_files/figure-latex/unnamed-chunk-5-1.pdf}

\hypertarget{dado-lo-observado-quitamos-los-datos-antes-del-1993}{%
\subsection{Dado lo observado quitamos los datos antes del
1993}\label{dado-lo-observado-quitamos-los-datos-antes-del-1993}}

\hypertarget{autos-promedio-de-co-a-travuxe9s-del-tiempo-con-el-arreglo}{%
\subsection{Autos \& promedio de CO a través del tiempo con el
arreglo}\label{autos-promedio-de-co-a-travuxe9s-del-tiempo-con-el-arreglo}}

\begin{verbatim}
## `geom_smooth()` using formula 'y ~ x'
\end{verbatim}

\includegraphics{exploratorio_files/figure-latex/unnamed-chunk-7-1.pdf}

\hypertarget{pruebas-de-hipuxf3tesis}{%
\section{Pruebas de hipótesis}\label{pruebas-de-hipuxf3tesis}}

\% Table created by stargazer v.5.2.3 by Marek Hlavac, Social Policy
Institute. E-mail: marek.hlavac at gmail.com \% Date and time: Tue, May
17, 2022 - 07:34:55 PM \% Requires LaTeX packages: dcolumn

\begin{table}[!htbp] \centering 
  \caption{Parámetros estimados para el modelo lineal en \eqref{eq:spec-malo}.} 
  \label{} 
\begin{tabular}{@{\extracolsep{5pt}}lD{.}{.}{-3} } 
\\[-1.8ex]\hline 
\hline \\[-1.8ex] 
 & \multicolumn{1}{c}{\textit{Dependent variable:}} \\ 
\cline{2-2} 
\\[-1.8ex] & \multicolumn{1}{c}{log(MEAN\_CONT)} \\ 
\hline \\[-1.8ex] 
 log(POBLACION) & -27.397^{***}$ $(-33.895$, $-20.898) \\ 
  log(EST\_V) & 0.995^{*}$ $(-0.010$, $2.000) \\ 
  log(AUTOMOVILES) & -0.166$ $(-0.485$, $0.153) \\ 
  Constant & 435.315^{***}$ $(338.000$, $532.630) \\ 
 \hline \\[-1.8ex] 
Observations & \multicolumn{1}{c}{25} \\ 
R$^{2}$ & \multicolumn{1}{c}{0.981} \\ 
Adjusted R$^{2}$ & \multicolumn{1}{c}{0.979} \\ 
Residual Std. Error & \multicolumn{1}{c}{0.084 (df = 21)} \\ 
F Statistic & \multicolumn{1}{c}{365.822$^{***}$ (df = 3; 21)} \\ 
\hline 
\hline \\[-1.8ex] 
\textit{Note:}  & \multicolumn{1}{r}{$^{*}$p$<$0.1; $^{**}$p$<$0.05; $^{***}$p$<$0.01} \\ 
\end{tabular} 
\end{table}

La para probar Heterocedasticidad se usa la prueba Breusch--Pagan. Esta
prueba usa de hipótesis nula: \(H_{0}:\) Hay homocedasticidad presente
en el modelo (los residuales se distribuyen con la misma varianza)
\emph{vs} \(H_{a}:\) Los residuales se dsitribuyen con varianzas
distintas, es decir hay heterocedasticidad en modelo.

Para esta prueba estadística usamos la función \texttt{bptest}. El
\(p\)-value de la prueba Breusch--Pagan es 0.194183. Es decir como es
menor a los niveles aceptables de significancia se rechaza la hipótesis
nula. Por lo tante se puede concluir que el modelo presenta
heterocedasticidad. Por ello se buscará hacer un modelo más robusto
corrigiendo esto.

Ahora para además se uso la prueba White para corroborar lo anterior y
el resultado es el siguiente.

Es importante notar que la heterocedasticidad si está presente en el
modelo ya que el \(p\)-value, 0.2444776, está por debajo de los niveles
de signifincia aceptados (es decir es menor al 5\% o 10\%).

Para corregir la heterocedasticidad se hará el cálculo de
``heteroskedasticity-robust standard errors'' con el tipo HC1. La matriz
de coviaranza robusta (mejorada) de las variables es la siguiente:

\% Table created by stargazer v.5.2.3 by Marek Hlavac, Social Policy
Institute. E-mail: marek.hlavac at gmail.com \% Date and time: Tue, May
17, 2022 - 07:34:55 PM \% Requires LaTeX packages: dcolumn

\begin{table}[!htbp] \centering 
  \caption{Matriz de covarianza robusta para las variables presentadas.} 
  \label{} 
\begin{tabular}{@{\extracolsep{5pt}} D{.}{.}{-3} D{.}{.}{-3} D{.}{.}{-3} D{.}{.}{-3} D{.}{.}{-3} } 
\\[-1.8ex]\hline 
\hline \\[-1.8ex] 
\multicolumn{1}{c}{} & \multicolumn{1}{c}{(Intercept)} & \multicolumn{1}{c}{log(POBLACION)} & \multicolumn{1}{c}{log(EST\_V)} & \multicolumn{1}{c}{log(AUTOMOVILES)} \\ 
\hline \\[-1.8ex] 
\multicolumn{1}{c}{(Intercept)} & 1,268.118 & -83.003 & 0.704 & 3.692 \\ 
\multicolumn{1}{c}{log(POBLACION)} & -83.003 & 5.442 & -0.087 & -0.237 \\ 
\multicolumn{1}{c}{log(EST\_V)} & 0.704 & -0.087 & 0.216 & -0.031 \\ 
\multicolumn{1}{c}{log(AUTOMOVILES)} & 3.692 & -0.237 & -0.031 & 0.017 \\ 
\hline \\[-1.8ex] 
\end{tabular} 
\end{table}

Con ello también se puede ver que las desviaciones estandar (robustas)
de las variables es la siguiente:

\% Table created by stargazer v.5.2.3 by Marek Hlavac, Social Policy
Institute. E-mail: marek.hlavac at gmail.com \% Date and time: Tue, May
17, 2022 - 07:34:55 PM \% Requires LaTeX packages: dcolumn

\begin{table}[!htbp] \centering 
  \caption{Desviaciones estándar robustas para las variables presentadas.} 
  \label{} 
\begin{tabular}{@{\extracolsep{5pt}} D{.}{.}{-3} D{.}{.}{-3} D{.}{.}{-3} D{.}{.}{-3} } 
\\[-1.8ex]\hline 
\hline \\[-1.8ex] 
\multicolumn{1}{c}{(Intercept)} & \multicolumn{1}{c}{log(POBLACION)} & \multicolumn{1}{c}{log(EST\_V)} & \multicolumn{1}{c}{log(AUTOMOVILES)} \\ 
\hline \\[-1.8ex] 
35.611 & 2.333 & 0.465 & 0.132 \\ 
\hline \\[-1.8ex] 
\end{tabular} 
\end{table}

Por último con esta matriz de covarianzas robusta se puede calcular como
quedarían las estimaciones del modelo:

\% Table created by stargazer v.5.2.3 by Marek Hlavac, Social Policy
Institute. E-mail: marek.hlavac at gmail.com \% Date and time: Tue, May
17, 2022 - 07:34:55 PM \% Requires LaTeX packages: dcolumn

\begin{table}[!htbp] \centering 
  \caption{Estimadores del modelo calculados con las covarianzas robustas} 
  \label{} 
\begin{tabular}{@{\extracolsep{5pt}}lD{.}{.}{-3} } 
\\[-1.8ex]\hline 
\hline \\[-1.8ex] 
 & \multicolumn{1}{c}{\textit{Dependent variable:}} \\ 
\cline{2-2} 
\\[-1.8ex] & \multicolumn{1}{c}{ } \\ 
\hline \\[-1.8ex] 
 log(POBLACION) & -27.397^{***}$ $(-31.969$, $-22.825) \\ 
  log(EST\_V) & 0.995^{**}$ $(0.083$, $1.906) \\ 
  log(AUTOMOVILES) & -0.166$ $(-0.424$, $0.092) \\ 
  Constant & 435.315^{***}$ $(365.520$, $505.111) \\ 
 \hline \\[-1.8ex] 
\hline 
\hline \\[-1.8ex] 
\textit{Note:}  & \multicolumn{1}{r}{$^{*}$p$<$0.1; $^{**}$p$<$0.05; $^{***}$p$<$0.01} \\ 
\end{tabular} 
\end{table}

Se puede ver de manera clara que la variable del número de estaciones de
transporte verde continua siendo no significativa, lo cual es de
esperarse ya que el modelo original no presenta problemas de
heterocedasticidad ni autocorrelación. Como no tiene estos problemas,
estas ``correcciones'' no corrijen nada en realidad.

Se usó el siguiente correlograma para observar si el modelo puede tener
autocorrelación:
\includegraphics{exploratorio_files/figure-latex/unnamed-chunk-14-1.pdf}

La prueba que se usó para detectar al autocorrelación es la
Breusch-Godfrey. En ella la hipótesis nula es
\(H_{0}: No autocorrelación\) y los resultados son los siguientes

El p-value es el siguiente: 0.4755028 es mayor a un nivel de
significancia del 10\% por lo que no se rechaza \(H_{0}\), es decir no
hay autocorrelación.

Por último se probará por multicolinealidad con el uso del índice de
condición, este es el siguiente para el modelo:

\% Table created by stargazer v.5.2.3 by Marek Hlavac, Social Policy
Institute. E-mail: marek.hlavac at gmail.com \% Date and time: Tue, May
17, 2022 - 07:34:55 PM \% Requires LaTeX packages: dcolumn

\begin{table}[!htbp] \centering 
  \caption{Números de condición para la matriz de datos.} 
  \label{} 
\begin{tabular}{@{\extracolsep{5pt}}lD{.}{.}{-3} D{.}{.}{-3} D{.}{.}{-3} D{.}{.}{-3} D{.}{.}{-3} } 
\\[-1.8ex]\hline 
\hline \\[-1.8ex] 
Statistic & \multicolumn{1}{c}{N} & \multicolumn{1}{c}{Mean} & \multicolumn{1}{c}{St. Dev.} & \multicolumn{1}{c}{Min} & \multicolumn{1}{c}{Max} \\ 
\hline \\[-1.8ex] 
Eigenvalue & 4 & 1.000 & 2.000 & 0.00000 & 4.000 \\ 
Condition Index & 4 & 2,267.241 & 4,248.640 & 1.000 & 8,636.882 \\ 
intercept & 4 & 0.250 & 0.500 & 0.000 & 1.000 \\ 
log(POBLACION) & 4 & 0.250 & 0.500 & 0.000 & 1.000 \\ 
log(EST\_V) & 4 & 0.250 & 0.361 & 0.00000 & 0.767 \\ 
log(AUTOMOVILES) & 4 & 0.250 & 0.321 & 0.00000 & 0.712 \\ 
\hline \\[-1.8ex] 
\end{tabular} 
\end{table}

Se puede observar que (se requiere que dos o más variables tengan un
condition index mayor a 30, no entendí como leer esto pero

Además podemos observar que el Factor de Inflación de la Varianza (FIV)
es el siguiente:

\% Table created by stargazer v.5.2.3 by Marek Hlavac, Social Policy
Institute. E-mail: marek.hlavac at gmail.com \% Date and time: Tue, May
17, 2022 - 07:34:55 PM \% Requires LaTeX packages: dcolumn

\begin{table}[!htbp] \centering 
  \caption{} 
  \label{} 
\begin{tabular}{@{\extracolsep{5pt}}lD{.}{.}{-3} D{.}{.}{-3} D{.}{.}{-3} D{.}{.}{-3} D{.}{.}{-3} } 
\\[-1.8ex]\hline 
\hline \\[-1.8ex] 
Statistic & \multicolumn{1}{c}{N} & \multicolumn{1}{c}{Mean} & \multicolumn{1}{c}{St. Dev.} & \multicolumn{1}{c}{Min} & \multicolumn{1}{c}{Max} \\ 
\hline \\[-1.8ex] 
Tolerance & 3 & 0.117 & 0.080 & 0.061 & 0.208 \\ 
VIF & 3 & 11.178 & 5.859 & 4.805 & 16.332 \\ 
\hline \\[-1.8ex] 
\end{tabular} 
\end{table}

\% Error: Unrecognized object type.

Usamos una gráfica de correlacion entre variables para ver cuales
variables podrían estar causando la multicolinealidad

\includegraphics{exploratorio_files/figure-latex/unnamed-chunk-19-1.pdf}

Podemos ver que la cantidad de automóviles y la Población están
áltamente correlacionados. Estas variables podrían ser (en conjunto) las
responsables del problema grave de multicolinealidad que observamos.
Para arreglarlo nos deshacemos de la variable de población. Simplemente
porque tiene más sentido mantener en el modelo la cantidad de
automóviles, porque son las fuentes de contaminación.

\hypertarget{especificaciuxf3n-del-modelo-de-regresiuxf3n}{%
\section{Especificación del modelo de
regresión}\label{especificaciuxf3n-del-modelo-de-regresiuxf3n}}

\[
\ln(\text{CO}_{t}) = \beta_0 + \beta_1 \ln(\text{EST\_V}_{t})  + \beta_2 \ln(\text{AUTOMOVILES}_{t})
\]

En R lo espcificamos así:

\begin{Shaded}
\begin{Highlighting}[]
\NormalTok{spec }\OtherTok{\textless{}{-}} \FunctionTok{log}\NormalTok{(MEAN\_CONT) }\SpecialCharTok{\textasciitilde{}} \FunctionTok{log}\NormalTok{(EST\_V) }\SpecialCharTok{+} \FunctionTok{log}\NormalTok{(AUTOMOVILES)}
\end{Highlighting}
\end{Shaded}

\hypertarget{estimaciuxf3n-del-modelo}{%
\section{Estimación del modelo}\label{estimaciuxf3n-del-modelo}}

\hypertarget{estimaciuxf3n-ols}{%
\subsection{Estimación OLS}\label{estimaciuxf3n-ols}}

\% Table created by stargazer v.5.2.3 by Marek Hlavac, Social Policy
Institute. E-mail: marek.hlavac at gmail.com \% Date and time: Tue, May
17, 2022 - 07:34:55 PM \% Requires LaTeX packages: dcolumn

\begin{table}[!htbp] \centering 
  \caption{Estimación del modelo modificado para eliminar multicolinealidad.} 
  \label{} 
\begin{tabular}{@{\extracolsep{5pt}}lD{.}{.}{-3} } 
\\[-1.8ex]\hline 
\hline \\[-1.8ex] 
 & \multicolumn{1}{c}{\textit{Dependent variable:}} \\ 
\cline{2-2} 
\\[-1.8ex] & \multicolumn{1}{c}{log(MEAN\_CONT)} \\ 
\hline \\[-1.8ex] 
 log(EST\_V) & -1.085$ $(-2.849$, $0.678) \\ 
  log(AUTOMOVILES) & -1.296^{***}$ $(-1.644$, $-0.947) \\ 
  Constant & 25.199^{***}$ $(19.627$, $30.772) \\ 
 \hline \\[-1.8ex] 
Observations & \multicolumn{1}{c}{25} \\ 
R$^{2}$ & \multicolumn{1}{c}{0.920} \\ 
Adjusted R$^{2}$ & \multicolumn{1}{c}{0.913} \\ 
Residual Std. Error & \multicolumn{1}{c}{0.169 (df = 22)} \\ 
F Statistic & \multicolumn{1}{c}{126.802$^{***}$ (df = 2; 22)} \\ 
\hline 
\hline \\[-1.8ex] 
\textit{Note:}  & \multicolumn{1}{r}{$^{*}$p$<$0.1; $^{**}$p$<$0.05; $^{***}$p$<$0.01} \\ 
\end{tabular} 
\end{table}

\hypertarget{resultados}{%
\subsection{Resultados}\label{resultados}}

La estimación es idéntica, pero se ve que si hay correlaciones entre
variables por la matriz de correlaciones.

\begin{itemize}
\tightlist
\item
  Está bien hecho?
\item
  Son malas noticias?
\end{itemize}

\end{document}
